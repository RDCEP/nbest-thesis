\documentclass[draftthesis]{style/neiuthesis} \usepackage{natbib}

\ifpdf  
    \pdfinfo { /Title  (Best MA G\&ES NEIU Thesis)
               /Creator (TeX)
               /Producer (pdfTeX)
               /Author (Neil Best nbest@alum.mit.edu)
               /CreationDate (D:YYYYMMDDhhmmss)  %format D:YYYYMMDDhhmmss
               /ModDate (D:YYYYMMDDhhmm)
               /Subject (Land Use / Land Cover)
               /Keywords (agriculture, United States) }
    \pdfcatalog { /PageMode (/UseOutlines)
                  /OpenAction (fitbh)  }
\fi

\begin{document} \bibliographystyle{chicago}

\title{Synthesis of a complete land use / land cover data set for the
conterminous United States emphasizing accuracy in area and
distribution of agricultural activity} \author{Neil A. Best}
\degreeyear{December~2010}

\othermasters{Master of Arts}{M.A.}  \department{Geography \&
Environmental Studies}

\maketitle

\frontmatter

\begin{abstract} This paper presents an effort to produce a new land
cover data set for the conterminous United States that augments
available agricultural land use data with other uses and covers to
create a complete landscape characterization.  We start with the data
set described in Ramankutty, et al., 2008, which improves on the
spatial distributions of agricultural land indicated by the MODIS Land
Cover Type and Global Land Cover 2000 data products on which it is
based by incorporating agricultural census data as a ground truth
constraint.  However, it provides no information regarding areas which
were judged to have been misclassified.  We present a method for
reconciling the Ramankutty data with the MODIS Land Cover Type map for
2001 and aspects of the higher-resolution 2001 National Landcover
Database.  This result is subsequently merged with the data from
Monfreda, et al., 2008 in order to further disaggregate cropland into
commodity sub-classes.  We describe a prototype economic land use
change model driven by land conversion costs, crop yield expectations,
and climate change scenarios that requires this data for
initialization.  We examine trends in other data sets to assess the
accuracy of change in agricultural land area expressed in the MODIS
time series after 2001.  This effort points a way forward to the
eventual production of a global, annual time series of land cover/
land use maps featuring explicit disaggregation of croplands by
commodity to be used in economic modeling of global agricultural
production, trade, and consumption.
\end{abstract}

\chapter*{Acknowledgements}

This work was made possible through the support of my employer, the
Computation Institute, University of Chicago, and its director,
Dr. Ian Foster.


\tableofcontents \listoftables \listoffigures

%% Create a List of Abbreviations. The left column %% is 1 inch wide
and left-justified
\chapter{List of Abbreviations}

\begin{symbollist*}
\item[MODIS] Moderate-resolution Imaging Spectroradiometer
\item[MLCT] MODIS Land Cover Type \citep{MLCT}
\item[175Crops2000] Harvested Area and Yields of 175 crops (M3-Crops
Data), \citep{Monfreda2008}
\item[Aglands2000] \citep{Ramankutty2008}
\item [NLCD] National Land-Cover Database, 2001 \citep{Homer2004}
\end{symbollist*}

%% Create a List of Symbols. The left column %% is 0.7 inch wide and
centered
\chapter{List of Symbols}

\begin{symbollist}[0.7in]
\item[$\tau$] Time taken to drink one cup of coffee.
\item[$\mu$g] Micrograms (of caffeine, generally).
\end{symbollist}

\mainmatter

There are some papers that say some important stuff
\citep{Ramankutty2008}


\chapter{Introduction}
\label{cha:introduction}

Recent years have seen a significant increase in the availability of
global land cover data sets inclding Global Land Cover 2000 (GLC2000),
MODIS Land Cover Type (MLCT), \marginpar{*fixme* \\ other global data
  sets} something from 1994 based on AVHRR, . . .  MLCT stands out
among these due to its spatial resolution, nominally 500m, and its
distinction as a time series rather than a snapshot.  Economic models
of land use and land conversion require information that describes a
complete, albeit simplified, description of land cover and
land-intensive econommic activity in order to produce meaningful
statements and predictions about the evolution of land use patterns.
``Complete'' in this context means that all cover types or uses for a
given portion of land area are assigned a category in the model.
However while MLCT does satisfy this condition of completeness it
presents two new complications that we must first address.

The first is that MLCT presents an embarassment of riches in terms of
detail.  Regardless of its classification accuracy, which is discussed
below, the 15-arcsecond resolution, nominally 500m, is simply too much
information to be able to run the economic models in a reasonable
amount of time even on world-class high-performance computing
platforms.  \marginpar{Joshua: is this a fair justification for
  reducing the resolution?}  A current standard resolution for global models of
many types and global data sets is 5-arcminutes, which is equivalent
to a 400:1 pixel count reduction.  Other data sets featured in this
analysis use this resolution which is convenient for formulation.

The second requirement for the new complete land cover data set that
we wish to produce is that it provide greater information regarding
agricultural activity.  MLCT presents a single class for cropland but
we wish to further disaggregate the areas of agricultural production
according to a few major commodities in order to incorporate greater
detail of agronomic and commercial factors into the models.  As we
will see, \citet{Monfreda2008} provides a wealth of data in this
regard by harvested area and yield for 175 crops globally, but does
not provide a complete land cover description.  



\chapter{Analysis}
\label{cha:analysis}

The MLCT indicates 368.7 Ma (149.2 Mha) of cropland in the cUSA in 2001.  Assuming that 50\% of the cropland/natural vegetation mosaic is additional cropland area gives and additional 118.1 Ma (47.8 Mha) of agricultural land.  This gives a total area of  486.8 Ma (197.0 Mha) of total area directly associated with agricultural activity according to the IGBP classification used in the MLCT.

Aglands2000 indicates roughly 446 Ma (181 Mha) of cropland.

Pasture indicated by Aglands2000 appears to be a broader classification than that of the NLCD's pasture class because much of the grazing land east of the Mississippi river counted in the Aglands2000 pasture map is absent in the NLCD pasture class.

Due to its greater resolution (30m) the NLCD is better suited at discerning developed areas in rural landscapes ranging from rural roads to farmsteads to small communities that do not show up in the MLCT data.  There is a total area of roughly 74 Ma (30 Mha) of development remaining after subtracting the MLCT urban class from all developed classes in the NLCD where the NLCD shows greater development after they have both been aggregated to the 5-arcmin grid.  Applying this area as an offset to the cropland area in Aglands2000 brings us closer to the expected acreage under cultivation in 2001, although this assumes that all of that development intersects with MLCT cropland area.

% r.mapcalc nlcd_rural_dev@cusa='max(nlcd_urban_5min@nlcd-2001_urban_As00_5min@cusa,0)'


\chapter{Conclusions}

We conclude that graduate students like coffee.

\appendix*

%\include{appendix}

\backmatter

\bibliography{thesis}

\end{document}