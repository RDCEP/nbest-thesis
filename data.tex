
\chapter{Data}
\label{cha:data}

This chapter presents some summary descriptions of the various data
sets that are relevant to this analysis and further discussion on how
they were manipulated.

The general approach with the classified land cover data sets (MLCT,
NLCD, CDL) is to reclassify their categories and aggregate the new
classification to the 5-arcmin grid.  The purpose of the
reclassification is to reduce the number of classes and have a uniform
set of classes across data sets.  The challenge in this is that
classification defintions are sometimes subtly different which makes
direct comparison across data sets difficult.  In this process we
convert classified maps whose pixels have discreet values to a stack
of maps, one map per class, whose pixels have real number values on
the interval [0,1] and are constrained to sum to unity for each pixel
through the stack.

\missingfigure{Generate a summary table of data sets (raster/tabular, resolution, citation)}

\section{MODIS Land Cover Type (MLCT)}
\label{sec:mlct}

\subsection{Reclassification}
\label{sec:mlct-reclass}

\missingfigure{MLCT reclassification table}

\subsection{Aggregation}
\label{sec:mlct-aggr}

This product is pixelated at 500m resolution and uses the 17 IGBP land
cover classes. The dataset consists of a primary classification, along
with a measure of confidence up to 100\%, and a secondary
classification.  According to the MODIS Land Cover Group, the primary
class covers at least roughly 50-60\% of a given pixel $x$, and this
percent is almost certainly a monotonically increasing function of the
confidence measure $c$.  \todo{cite email from Friedl}.  For the
purposes of this analysis we assume that this dependence is
linear. Thus, for the primary and secondary cover types in a pixel:

$$
A_p(x) = A_{min} + (1 - A_{min}) c(x)
$$
$$
A_s(x) = 1 - A_p(x)
$$

where $0.50 \le A_{min} \le 0.60$ is primarily chosen based on an
interpretation of $c$.  Given that there are only a handful of
examples of $c < 0.20$ (see figure XX for an example) \todo{crossref
  histograms showing confidence distribution}, setting $A_{min} =
0.50$ is both rational and prudent.  Certainly for a classification to
be considered the primary it must represent a bare majority of the
area covered by that pixel, and the distributions of confidences
indicate that the vast majority of pixels contain greater than 60\% of
their area in the primary under the rubric described above. The
equations are simplified as follows by assuming this value for
$A_{min}$.

$$
A_p(x) = \dfrac{1 + c}{2}
$$
$$
A_s(x) = 1 - A_p(x) = \dfrac{1-c}{2}
$$


\section{Agricultural Lands in the Year 2000 (Aglands2000)}
\label{sec:aglands2000}

\section{Major Land Uses (MLU)}
\label{sec:mlu}


\section{National Land-cover Database 2001 (NLCD)}
\label{sec:nlcd}

\subsection{Reclassification}
\label{sec:nlcd-reclass}

\missingfigure{NLCD reclassification table}

\subsection{Aggregation}
\label{sec:nlcd-aggr}


\section{Cropland Data Layer (CDL)}
\label{sec:cdl}

\missingfigure{Table or chart showing CDL covereage for various years}

\subsection{Reclassification}
\label{sec:cdl-reclass}

\missingfigure{CDL reclassification table}

\subsection{Aggregation}
\label{sec:cdl-aggr}


\section{Harvested Area and Yields of 175 Crops (175crops2000)}
\label{sec:175crops2000}

\missingfigure{Table of crops and types reproduced from \citep{Monfreda2008}}

\missingfigure{Summary table of crop aggregations for our model}

\todo{Adress issue of smaller land mask for 175crops2000 and Aglands2000}


\subsection{Reclassification}
\label{sec:cdl-reclass}

\missingfigure{CDL reclassification table}

\subsection{Aggregation}
\label{sec:cdl-aggr}

%%% Local Variables: 
%%% mode: latex
%%% TeX-master: "thesis"
%%% End: 
