
\chapter{Introduction}
\label{cha:introduction}

\section{Objective}
\label{sec:objective}

Recent years have seen a significant increase in the availability of
global land cover data sets inclding the UMD Global Land Cover
Classification product of 1998 \citep{Hansen2000}, Global Land Cover
2000 (GLC2000) \todo{GLC2000 citation}, 
MODIS Land Cover Type (MLCT) \todo{MLCT citation}.  MLCT stands out
among these due to its spatial resolution, nominally 500m, and its
distinction as a time series rather than a snapshot.  Economic models
of land use and land conversion require information that describes a
complete, albeit simplified, description of land cover and
land-intensive econommic activity in order to produce meaningful
statements and predictions about the evolution of land use patterns.
``Complete'' in this context means that all cover types or uses for a
given portion of land area are assigned a category in the model.
However while MLCT does satisfy this condition of completeness it
presents two new complications that we must first address.

The first is that MLCT presents an embarassment of riches in terms of
detail.  Regardless of its classification accuracy, which is discussed
below, the 15-arcsecond resolution, nominally 500m, is simply too much
information to be able to run the economic models in a reasonable
amount of time even on world-class high-performance computing
platforms.  A current standard resolution for global models of many
types and global data sets is 5-arcminutes, which is equivalent to a
400:1 pixel count reduction.  Other data sets featured in this
analysis use this resolution which is convenient for formulation.

The second requirement for the new complete land cover data set that
we wish to produce is that it provide greater information regarding
agricultural activity.  MLCT presents a single class for cropland but
we wish to further disaggregate the areas of agricultural production
according to a few major commodities in order to incorporate greater
detail of agronomic and commercial factors into the models.  As we
will see, \citet{Monfreda2008} provides a wealth of data in this
regard by harvested area and yield for 175 crops globally, but does
not provide a complete land cover description.  

\section{Tools}
\label{sec:tools}

A secondary objective of this paper is to demonstrate the capabilities
of a set of open-source geospatial, analytical, and publishing
software that includes \href{http://www.gdal.org/}{GDAL},
\href{http://grass.osgeo.org/}{GRASS} \citep{GRASS},
\href{http://www.r-project.org/}{R} \citep{R} , and
\href{http://www.latex-project.org/}{\LaTeX} \citep{Lamport1994} .
The last two members of this list are bridged by
\href{http://www.stat.uni-muenchen.de/~leisch/Sweave/}{Sweave}
\citep{Leisch2002} which allows embedding of analytical code written
in the R language within a \LaTeX document so that one step towards
producing a publication-quality PDF is running the analysis and
injecting its results directly into the content of the paper,
including tables, charts, and maps.  The underlying analysis code will
appear as an appendix.  This is a demonstration of reproducible
research as described in \citet{Gentleman2007}.
