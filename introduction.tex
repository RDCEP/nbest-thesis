
\chapter{Introduction}
\label{cha:introduction}

Recent years have seen a significant increase in the availability of
global land cover data sets inclding Global Land Cover 2000 (GLC2000),
MODIS Land Cover Type (MLCT), \marginpar{*fixme* \\ other global data
  sets} something from 1994 based on AVHRR, . . .  MLCT stands out
among these due to its spatial resolution, nominally 500m, and its
distinction as a time series rather than a snapshot.  Economic models
of land use and land conversion require information that describes a
complete, albeit simplified, description of land cover and
land-intensive econommic activity in order to produce meaningful
statements and predictions about the evolution of land use patterns.
``Complete'' in this context means that all cover types or uses for a
given portion of land area are assigned a category in the model.
However while MLCT does satisfy this condition of completeness it
presents two new complications that we must first address.

The first is that MLCT presents an embarassment of riches in terms of
detail.  Regardless of its classification accuracy, which is discussed
below, the 15-arcsecond resolution, nominally 500m, is simply too much
information to be able to run the economic models in a reasonable
amount of time even on world-class high-performance computing
platforms.  \marginpar{Joshua: is this a fair justification for
  reducing the resolution?}  A current standard resolution for global models of
many types and global data sets is 5-arcminutes, which is equivalent
to a 400:1 pixel count reduction.  Other data sets featured in this
analysis use this resolution which is convenient for formulation.

The second requirement for the new complete land cover data set that
we wish to produce is that it provide greater information regarding
agricultural activity.  MLCT presents a single class for cropland but
we wish to further disaggregate the areas of agricultural production
according to a few major commodities in order to incorporate greater
detail of agronomic and commercial factors into the models.  As we
will see, \citet{Monfreda2008} provides a wealth of data in this
regard by harvested area and yield for 175 crops globally, but does
not provide a complete land cover description.  

